%!TEX root = diss.tex
\chapter{Glossary}

% use \acrodef to define an acronym, but no listing
\acrodef{UI}{user interface}

\begin{acronym}[VAST]
\acro{ANOVA}[ANOVA]{Analysis of Variance\acroextra{, a set of
  statistical techniques to identify sources of variability between groups.}}
\acro{API}{application programming interface}
\acro{TLX}[TLX]{Task Load Index\acroextra{, an instrument for gauging
  the subjective mental workload experienced by a human in performing
  a task.}}
\acro{UML}{Unified Modelling Language\acroextra{, a visual language
    for modelling the structure of software artefacts.}}
\acro{W3C}[W3C]{\acroextra{the }World Wide Web Consortium\acroextra{,
    the standards body for web technologies.}}
\acro{XML}{Extensible Markup Language}
\end{acronym}

% You can also use something like the following to only define acronyms
% but without explictly creating a glossary
% 
% \newacro{ANOVA}[ANOVA]{Analysis of Variance\acroextra{, a set of
%   statistical techniques to identify sources of variability between groups.}}
% \newacro{API}[API]{application programming interface}
% \newacro{GOMS}[GOMS]{Goals, Operators, Methods, and Selection\acroextra{,
%   a framework for usability analysis.}}
% \newacro{TLX}[TLX]{Task Load Index\acroextra{, an instrument for gauging
%   the subjective mental workload experienced by a human in performing
%   a task.}}
% \newacro{UI}[UI]{user interface}
% \newacro{UML}[UML]{Unified Modelling Language}
% \newacro{W3C}[W3C]{World Wide Web Consortium}
% \newacro{XML}[XML]{Extensible Markup Language}
