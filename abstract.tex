%% The following is a directive for TeXShop to indicate the main file
%%!TEX root = template.tex
\chapter{Abstract}

This document provides brief instructions for using the \class{ubcdiss}
class to write a \acs{UBC}-conformant dissertation in \LaTeX.  This
document is itself written using the \class{ubcdiss} class and is
intended to serve as an example of writing a dissertation in \LaTeX.
This document has embedded \acp{URL} and is intended to be viewed
using a computer-based \ac{PDF} reader.

Suggestions for improvements should be sent to Brian de Alwis at
\texttt{briandealwis@gmail.com}.

% Note: these must also be defined in glossary.tex
\newacro{UBC}{The University of British Columbia}
\newacro{FoGS}[FoGS]{Faculty of Graduate Studies}
\newacro{PDF}{Portable Document Format}
\newacro{URL}{Uniform Resource Locator}

Note: Abstracts should generally try to avoid using acronyms.  If you
must, use the \verb+acronym+ package's \verb+\newacro+ to define
local versions here; these definitions will be overwritten by those
in \filename{glossary.tex}.  See the \filename{glossary.tex} file for
examples of defining elements for a glossary.
For example, the source for this page added the following definition:
\begin{verbatim}
    \newacro{UBC}{The University of British Columbia}
    \newacro{FoGS}{Faculty of Graduate Studies}
    \newacro{PDF}{Portable Document Format}
    \newacro{URL}{Uniform Resource Locator}
\end{verbatim}
so as to refer to `\ac{UBC}' and `\ac{FoGS}'.

Note: at \ac{UBC}, both the \ac{FoGS} Ph.D. defence programme and the
Library's online submission system require that abstracts to be 350
words or less.
