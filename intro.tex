%% The following is a directive for TeXShop to indicate the main file
%%!TEX root = template.tex

\chapter{Introduction}
\label{ch:Introduction}

\begin{epigraph}
    \emph{If I have seen farther it is by standing on the shoulders of
    Giants.} ---~Sir Isaac Newton (1855)
\end{epigraph}

This document provides a quick set of instructions for using the
\class{ubcdiss} class to write a dissertation in \LaTeX. 
Unfortunately this document cannot provide an introduction to using
\LaTeX.  The classic reference for learning \LaTeX\ is
\citeauthor{lamport-1994-ladps}'s
book~\cite{lamport-1994-ladps}.  There are also many freely-available
tutorials online;
\webref{http://www.andy-roberts.net/misc/latex/}{Andy Roberts' online
    \LaTeX\ tutorials}
seems to be excellent.
The source code for this docment, however, is intended to serve as
an example for creating a \LaTeX\ version of your dissertation.

We start by discussing organizational issues, such as splitting
your dissertation into multiple files, in
\autoref{sec:SuggestedThesisOrganization}.
We then cover the ease of managing cross-references in \LaTeX\ in
\autoref{sec:CrossReferences}.
We cover managing and using bibliographies with \BibTeX\ in
\autoref{sec:BibTeX}. 
We briefly describe typesetting attractive tables in
\autoref{sec:TypesettingTables}.
We briefly describe including external figures and using
special symbols in \LaTeX\ in \autoref{sec:Graphics}.
As it is often useful to track different versions of your dissertation,
we discuss revision control further in
\autoref{sec:DissertationRevisionControl}. 
We conclude with pointers to additional sources of information in
\autoref{sec:Conclusions}.

%%%%%%%%%%%%%%%%%%%%%%%%%%%%%%%%%%%%%%%%%%%%%%%%%%%%%%%%%%%%%%%%%%%%%%
\section{Suggested Thesis Organization}
\label{sec:SuggestedThesisOrganization}

The \acs{UBC} \acf{FoGS} specifies a particular arrangement of the
components forming a thesis.\footnote{See
    \url{http://www.grad.ubc.ca/students/thesis/index.asp?menu=005,000,000,000}}
This template reflects that arrangement.

In terms of writing your thesis, the recommended best practice for
organizing large documents in \LaTeX\ is to place each chapter in
a separate file.  These chapters are then included from the main
file through the use of \verb+\include{file}+.  A thesis might
be described as six files such as \file{intro.tex},
\file{relwork.tex}, \file{model.tex}, \file{eval.tex},
\file{discuss.tex}, and \file{concl.tex}.


\subsection{Changing Page Widths and Heights}

The \class{ubcdiss} class is based on the standard \LaTeX\ \class{book}
class that selects a line-width to carry approximately 66~characters
per line.  This character density is claimed to have a pleasing
appearance and also supports more rapid
reading~\cite{bringhurst-2002-teots}.  I would recommend that you
not change the line-widths!

Some students are unfortunately saddled with misguided supervisors
or committee members whom believe that documents should have the
narrowest margins possible.  The \package{geometry} package is
helpful in such cases.  Using this package is as simple as:
\begin{lstlisting}
    \usepackage[margin=1.25in,top=1.25in,bottom=1.25in]{geometry}
\end{lstlisting}
You should check the package's documentation for more complex uses.

There are occasions where it becomes necessary to make temporary
changes to the page width, such as to accomodate a larger formula. 
The \packagemisc{chngpage} package provides an \env{adjustwidth}
environment that does just this.  For example:
\begin{lstlisting}
    % Expand left and right margins by 0.75in
    \begin{adjustwidth}{-0.75in}{-0.75in}
    % We may need to adjust the perceived column
    % width for LaTeX to get with it.
    \addtolength{\columnwidth}{1.5in}
    \[ an extra long math formula \]
    \end{adjustwidth}
\end{lstlisting}

%%%%%%%%%%%%%%%%%%%%%%%%%%%%%%%%%%%%%%%%%%%%%%%%%%%%%%%%%%%%%%%%%%%%%%
\section{Cross-References}
\label{sec:CrossReferences}

\LaTeX\ make managing cross-references easy, and the \package{hyperref}
package's\ \verb+\autoref+ command\footnote{%
    The \package{hyperref} package is included by default in this
    template.}
makes it easier still. 

A thing to be cross-referenced, such as a section, figure, or equation,
is \emph{labelled} using a unique, user-provided identifier, defined
using the \verb+\label+ command.  
The thing is referenced elsewhere using the \verb+\autoref+ command.
For example, this section was defined using:
\begin{lstlisting}
    \section{Cross-References}
    \label{sec:CrossReferences}
\end{lstlisting}
References to this section are done as follows:
\begin{lstlisting}
    We then cover the ease of managing cross-references in \LaTeX\
    in \autoref{sec:CrossReferences}.
\end{lstlisting}
\verb+\autoref+ takes care of determining \emph{type} of the 
thing being referenced, so \verb+\autoref{sec:CrossReferences}+
is replaced with the text ``\autoref{sec:CrossReferences}.''

A identifer is a simple sequence of characters, numbers, digits,
and some punctuation marks such as ``:'' and ``--''; there should
be no spaces.  A consistent key format simiplifies remembering how
to make references.  This document uses a prefix to indicate the
type of the thing being referenced, such as \texttt{sec} for sections,
\texttt{fig} for figures, \texttt{tbl} for tables, and \texttt{eqn}
for equations.

For further details on defining the text used for the thing types,
see the \class{hyperref} documentation.


%%%%%%%%%%%%%%%%%%%%%%%%%%%%%%%%%%%%%%%%%%%%%%%%%%%%%%%%%%%%%%%%%%%%%%
\section{Managing Bibliographies with \BibTeX}
\label{sec:BibTeX}

One of the primary benefits of using \LaTeX\ is its companion program,
\BibTeX, for managing bibliographies and citations.  Managing
bibliographies has three parts: (i) describing references,
(ii)~citing references, and (iii)~formatting cited references.

\subsection{Describing References}

\BibTeX\ defines a standard format for recording details about a
reference.  These references are recorded in a file with a
\file{.bib} extension.  \BibTeX\ supports a broad range of
references, such as books, articles, items in a conference proceedings,
chapters, technical reports, manuals, dissertations, and unpublished
manuscripts. 
A reference may include attributes such as the authors,
the title, the page numbers, the \ac{DOI}, or a \ac{URL}.  A reference
can also be augmented with personal attributes, such as a rating,
notes, or keywords.

Each reference must be described by a unique \emph{key}.\footnote{%
    Note that the citation keys are different from the reference
    identifiers as described in \autoref{sec:CrossReferences}.}
A key is a simple sequence of characters, numbers, digits, and some
punctuation marks such as ``:'' and ``--''; there should be no spaces. 
A consistent key format simiplifies remembering how to make references. 
For example:
\begin{quote}
   \fbox{\emph{lowercased-last-name-of-first-author}} \texttt{-}
    \fbox{\emph{year}} \texttt{-} \fbox{\emph{contracted-title}}
\end{quote}
Then G. Kiczales \etal's article, \emph{Aspect-Oriented Programming},
published in 1997, would have key \texttt{kiczales-1997-aop}.

An example of a \BibTeX\ \file{.bib} file is included as
\file{biblio.bib}.  A description of the format a \file{.bib}
file is beyond the scope of this document.  We instead encourage
you to use one of the several reference managers that support the
\BibTeX\ format such as
\webref{http://jabref.sourceforge.net}{JabRef} (multiple platforms) or
\webref{http://bibdesk.sourceforge.net}{BibDesk} (MacOS\,X only). 
These front ends are similar to reference manages such as
EndNote or RefWorks.


\subsection{Citing References}

Having described some references, we then need to cite them.  We
do this using a form of the \verb+\cite+ command.  For example:
\begin{lstlisting}
    \citet{kiczales-1997-aop} present examples of crosscutting 
    occurring in several languages.
\end{lstlisting}
When processed, the \verb+\citet+ will cause the paper's authors
and a standardized reference to the paper to be inserted in the
document, and will also include a formatted citation for the paper
in the bibliography.  For example:
\begin{quote}
    \citet{kiczales-1997-aop} present examples of crosscutting 
    occurring in several languages.
\end{quote}
There are several forms of the \verb+\cite+ command (provided
by the \package{natbib} package); see \autoref{tbl:natbib:cite}.
\begin{table}
\caption{Available \texttt{cite} variants.}
\label{tbl:natbib:cite}
\begin{tabular}{ll}\toprule
Variant & Result \\
\midrule
\verb+\citet+ & Textual citation: includes author (\eg \citet{kiczales-1997-aop}) \\
\verb+\citet*+ & Textual citation with unabbreviated author list \\
\verb+\citep+ & Parenthetical citation (\eg \citep{kiczales-1997-aop}) \\
\verb+\citep*+ & Parenthetical citation with unabbreviated author list \\
\verb+\citeauthor+ & Author only (\eg \citeauthor{kiczales-1997-aop}) \\
\verb+\citeyear+ & Year of citation (\eg \citeyear{kiczales-1997-aop}) \\
\bottomrule
\end{tabular}
\end{table}

\subsection{Formatting Cited References}

\BibTeX\ separates the citing of a reference from how the cited
reference is formatted for a bibliography, specified with the
\verb+\bibliographystyle+ command.  There are several varieties,
including \texttt{plainnat}, \texttt{abbrvnat}, and \texttt{unsrtnat}.

%%%%%%%%%%%%%%%%%%%%%%%%%%%%%%%%%%%%%%%%%%%%%%%%%%%%%%%%%%%%%%%%%%%%%%
\section{Typesetting Tables}
\label{sec:TypesettingTables}

Unfortunately \citet{lamport-1994-ladps} made one grievous mistake
in \LaTeX, the suggested manner for typesetting tables results
in abysmally ugly abominations.  These suggestions have been
perpetuated in most \LaTeX\ tutorials.
Fortunately this situation is easily rectified: simply ignore the
instructions on using horizontal and
vertical rules (specifically the use of \verb+\hline+ and \verb+|+) and
use the \package{booktabs} package instead.

The \package{booktabs} environment produces far more professional-looking
tables, and are the form used by most professionally-edited journals.
\autoref{tbl:natbib:cite} and \autoref{tbl:LaTeX:Symbols} are two
examples of tables typeset with the \package{booktabs} package.
The \package{booktabs} package provides three additional commands,
intended to replace the use of \verb+\hline+:
\verb+\toprule+ for the rule to appear at the top of the table,
\verb+\midrule+ for the middle rule following the table header,
and \verb+\bottomrule+ for the bottom-most at the end of the table.
These rules differ by their weight (thickness) and the spacing before
and after.
A table is typeset in the following manner:
\begin{lstlisting}
    \begin{table}
    \caption{The caption for the table}
    \label{tbl:label}
    \centering
    \begin{tabular}{cc}
    \toprule
    Header & Elements \\
    \midrule
    Row 1 & Row 1 \\
    Row 2 & Row 2 \\
    % ... and on and on ...
    Row N & Row N \\
    \bottomrule
    \end{tabular}
    \end{table}
\end{lstlisting}
See the \package{booktabs} documentation for advice in dealing with
special cases, such as subheading rules, introducing extra space
for divisions, and interior rules.

%%%%%%%%%%%%%%%%%%%%%%%%%%%%%%%%%%%%%%%%%%%%%%%%%%%%%%%%%%%%%%%%%%%%%%
\section{Figures, Graphics, and Special Characters}
\label{sec:Graphics}

Most beginners to \LaTeX\ find managing figures to be one of the
more difficult topics.  These difficulties arise from tussles of
control between the beginners and \LaTeX.  In \LaTeX's terms, a figure
is a \emph{floating element}, to be placed where it best fits.
The user is not expected to concern him/herself with the placement
of the figure.  Figures can be labelled, and when the text needs
to reference the figure, it should include a reference to the figure
using \verb+\autoref+ to reference the label. 
\autoref{fig:latex-affirmation} is an example of a figure.
\begin{figure}
    \centering
    % For the sake of this example, we'll just use text
    %\includegraphics[width=3in]{file}
    \Huge{\textsf{\LaTeX\ Rocks!}}
    \caption{Proof of \LaTeX's amazing abilities}
    \label{fig:latex-affirmation}   % label should change
\end{figure}
A figure is generally included as follows:
\begin{lstlisting}
    \begin{figure}
    \centering
    \includegraphics[width=3in]{file}
    \caption{A useful caption}
    \label{fig:fig-label}   % label should change
    \end{figure}
\end{lstlisting}
There are three items of note:
\begin{enumerate}
\item External files are included using the \verb+\includegraphics+
    command.  This command is defined by the \package{graphicx} package
    and can often natively import graphics from a variety of formats.
    The set of formats supported depends on your \TeX\ command processor.
    Both \texttt{pdflatex} and \texttt{xelatex}, for example, can
    import \textsc{gif}, \textsc{jpg}, and \textsc{pdf}.  The plain
    version of \texttt{latex} only supports \textsc{eps} files.

\item The \verb+\caption+ provides a caption to the figure.

\item The \verb+\label+ command provides for associating a unique, user-defined,
    and descriptive identifier to the figure.  The figure can be
    can be referenced elsewhere in the text with this identifier
    as described in \autoref{sec:CrossReferences}.
\end{enumerate}
See Keith Reckdahl’s excellent guide for more details,
\webref{http://www.ctan.org/tex-archive/info/epslatex.pdf}{\emph{Using
imported graphics in LaTeX2e}}.

\LaTeX\ is rather particular about the format of its input files,
with many common symbols appropriated for other purposes (\ie
``\verb+~$%^&_{}\+'') and others are automagically transformed into
typographically-preferred forms (\ie ``\verb+-`'+'') or something
completely different (\ie ``\verb+<>+'').
\autoref{tbl:LaTeX:Symbols} presents a list of common symbols and
their corresponding \LaTeX\ commands.  A more comprehensive list 
of symbols is available at:
\url{http://www.ctan.org/tex-archive/info/symbols/comprehensive/}
\begin{table}
\caption{Useful \LaTeX\ symbols}\label{tbl:LaTeX:Symbols}
\centering\begin{tabular}{cc}\toprule
\LaTeX & Result \\
\midrule
    \texttrademark & \verb+\texttrademark+ \\
    \textcopyright & \verb+\textcopyright+ \\
    \textregistered & \verb+\textregistered+ \\
    \textsection & \verb+\textsection+ \\
    \textdagger & \verb+\textdagger+ \\
    \textdaggerdbl & \verb+\textdaggerdbl+ \\
    \textless & \verb+\textless+ \\
    \textgreater & \verb+\textgreater+ \\
\bottomrule
\end{tabular}
\end{table}

%%%%%%%%%%%%%%%%%%%%%%%%%%%%%%%%%%%%%%%%%%%%%%%%%%%%%%%%%%%%%%%%%%%%%%
\section{Keeping Track of Older Versions with Revision Control}
\label{sec:DissertationRevisionControl}.

Software engineers have used \acf{RCS} to track changes to their
software systems for decades.  These systems record the changes to
the source code along with context as to why the change was required.
These systems also support examining and reverting to particular
revisions from their system's past.

An \ac{RCS} can be used to keep track of changes to things other
than source code, such as your dissertation.  For example, it can
be useful to know exactly which revision of your dissertation was
sent to a particular committee member.  Or to recover an accidentally
deleted file, or a badly modified image.  With a revision control
system, you can tag or annotate the revision of your dissertation
that was sent to your committee, or when you incorporated changes
from your supervisor.

Unfortunately current revision control packages are not yet targetted
to non-developers.  But the Subversion project's
\webref{http://tortoisesvn.net/docs/release/TortoiseSVN_en/}{TortoiseSVN}
has greatly simplified using the Subversion revision control system
for Windows users.  You should consult your local geek.

A simpler alternative strategy is to create a GoogleMail account
and periodically mail yourself zipped copies of your dissertation.

%%%%%%%%%%%%%%%%%%%%%%%%%%%%%%%%%%%%%%%%%%%%%%%%%%%%%%%%%%%%%%%%%%%%%%
\section{Conclusion}
\label{sec:Conclusions}

At this point, you should be ready to go.  Other handy web resources:
\begin{itemize}
\item \webref{http://www.ctan.org}{\ac{CTAN}} is \emph{the} comprehensive
    archive site for all things related to \TeX\ and \LaTeX.

\item The \webref{http://www.tug.org/tetex/tetex-texmfdist/doc/}{te\TeX\
    documentation guide} features a very handy list of the most useful
    packages for \LaTeX\ as found in \ac{CTAN}.

\item An online
    \webref{http://www.ctan.org/get/info/latex2e-help-texinfo/latex2e.html}{%
	reference to \LaTeX\ commands}.

\item The list of 
    \href{http://www.tex.ac.uk/cgi-bin/texfaq2html?label=interruptlist}{%
	Frequently Asked Questions about \TeX\ and \LaTeX}.

\item The
\webref{http://www.ctan.org/tex-archive/macros/latex/required/graphics/grfguide.pdf}{\texttt{color}}
    package, part of the graphics bundle, provides for changing text
    and background colours.

\item Chemists and those requiring chemical formula should definitely
    checkout the \package{mhchem} package.

\item If you're really keen, you might want to join the
    \webref{http://www.tug.org}{\TeX\ Users Group}.

\end{itemize}

\endinput

Any text after an \endinput is ignored.
You could put scraps here or things in progress.
